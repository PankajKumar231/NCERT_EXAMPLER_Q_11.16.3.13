\let\negmedspace\undefined
\let\negthickspace\undefined
\documentclass[journal,12pt,twocolumn]{IEEEtran}
\usepackage{cite}
\usepackage{amsmath,amssymb,amsfonts,amsthm}
\usepackage{algorithmic}
\usepackage{graphicx}
\usepackage{textcomp}
\usepackage{xcolor}
\usepackage{txfonts}
\usepackage{listings}
\usepackage{enumitem}
\usepackage{mathtools}
\usepackage{gensymb}
\usepackage[breaklinks=true]{hyperref}
\usepackage{tkz-euclide} % loads  TikZ and tkz-base
\usepackage{listings}
\usepackage{gvv}
%
%\usepackage{setspace}
%\usepackage{gensymb}
%\doublespacing
%\singlespacing

%\usepackage{graphicx}
%\usepackage{amssymb}
%\usepackage{relsize}
%\usepackage[cmex10]{amsmath}
%\usepackage{amsthm}
%\interdisplaylinepenalty=2500
%\savesymbol{iint}
%\usepackage{txfonts}
%\restoresymbol{TXF}{iint}
%\usepackage{wasysym}
%\usepackage{amsthm}
%\usepackage{iithtlc}
%\usepackage{mathrsfs}
%\usepackage{txfonts}
%\usepackage{stfloats}
%\usepackage{bm}
%\usepackage{cite}
%\usepackage{cases}
%\usepackage{subfig}
%\usepackage{xtab}
%\usepackage{longtable}
%\usepackage{multirow}
%\usepackage{algorithm}
%\usepackage{algpseudocode}
%\usepackage{enumitem}
%\usepackage{mathtools}
%\usepackage{tikz}
%\usepackage{circuitikz}
%\usepackage{verbatim}
%\usepackage{tfrupee}
%\usepackage{stmaryrd}
%\usetkzobj{all}
%    \usepackage{color}                                            %%
%    \usepackage{array}                                            %%
%    \usepackage{longtable}                                        %%
%    \usepackage{calc}                                             %%
%    \usepackage{multirow}                                         %%
%    \usepackage{hhline}                                           %%
%    \usepackage{ifthen}                                           %%
  %optionally (for landscape tables embedded in another document): %%
%    \usepackage{lscape}     
%\usepackage{multicol}
%\usepackage{chngcntr}
%\usepackage{enumerate}

%\usepackage{wasysym}
%\documentclass[conference]{IEEEtran}
%\IEEEoverridecommandlockouts
% The preceding line is only needed to identify funding in the first footnote. If that is unneeded, please comment it out.

\newtheorem{theorem}{Theorem}[section]
\newtheorem{problem}{Problem}
\newtheorem{proposition}{Proposition}[section]
\newtheorem{lemma}{Lemma}[section]
\newtheorem{corollary}[theorem]{Corollary}
\newtheorem{example}{Example}[section]
\newtheorem{definition}[problem]{Definition}
%\newtheorem{thm}{Theorem}[section] 
%\newtheorem{defn}[thm]{Definition}
%\newtheorem{algorithm}{Algorithm}[section]
%\newtheorem{cor}{Corollary}
\newcommand{\BEQA}{\begin{eqnarray}}
\newcommand{\EEQA}{\end{eqnarray}}
\newcommand{\define}{\stackrel{\triangle}{=}}
\theoremstyle{remark}
\newtheorem{rem}{Remark}

%\bibliographystyle{ieeetr}
\begin{document}
%

\bibliographystyle{IEEEtran}


\vspace{3cm}

\title{
\Huge \ Assignment3(Q_11.16.3.13)\\ NCERT EXAMPLER\\
\Huge\ Probability And Random Processes\\
\large\author{Pankaj Kumar \\EE22BTECH11040}
}





% make the title area
\maketitle

\newpage

%\tableofcontents

\bigskip

\renewcommand{\thefigure}{\theenumi}
\renewcommand{\thetable}{\theenumi}
%\renewcommand{\theequation}{\theenumi}

%\begin{abstract}
%%\boldmath
%In this letter, an algorithm for evaluating the exact analytical bit error rate  (BER)  for the piecewise linear (PL) combiner for  multiple relays is presented. Previous results were available only for upto three relays. The algorithm is unique in the sense that  the actual mathematical expressions, that are prohibitively large, need not be explicitly obtained. The diversity gain due to multiple relays is shown through plots of the analytical BER, well supported by simulations. 
%
%\end{abstract}
% IEEEtran.cls defaults to using nonbold math in the Abstract.
% This preserves the distinction between vectors and scalars. However,
% if the journal you are submitting to favors bold math in the abstract,
% then you can use LaTeX's standard command \boldmath at the very start
% of the abstract to achieve this. Many IEEE journals frown on math
% in the abs

% Note that keywords are not normally used for peerreview papers.
%\begin{IEEEkeywords}
%Cooperative diversity, decode and forward, piecewise linear
%\end{IEEEkeywords}



% For peer review papers, you can put extra information on the cover
% page as needed:
% \ifCLASSOPTIONpeerreview
% \begin{center} \bfseries EDICS Category: 3-BBND \end{center}
% \fi
%
% For peerreview papers, this IEEEtran command inserts a page break and
% creates the second title. It will be ignored for other modes.
%\IEEEpeerreviewmaket
\section{\huge Q11.16.3.13}
\large A bag contains 8 red and 5 white balls. Three balls are drawn at random. Find the
Probability that\\
(a) All the three balls are white\\
(b) All the three balls are red\\
(c) One ball is red and two balls are white\\
\solution
Let, \\
\[Y = {0,1}\]
\[0 : red \]
\[1 : white \]
To find the probabilities using the Probability Mass Function (PMF), we'll use the hypergeometric distribution\\
The PMF for the hypergeometric distribution is given by:

P(X = x) = [C(k, x) * C(N - k, n - x)]/C(N, n)

Where:

    N is the total number of balls in the bag\\
    k is the number of successful outcomes\\
    n is the number of draws\\
    x is the number of successful outcomes in your specific scenario\\
    C(a, b) represents the binomial coefficient, which is the number of ways to choose b items from a distinct items\\

\begin{align}
    
(a) All the three balls are white:\\
P(X = 0 red) = [C(8, 0) * C(5, 3)] / C(13, 3)\\

             = [(1) * (10)] / (286)\\

             = 10 / 286 ≈ 0.034965\\
(b) All the three balls are red:\\
P(X = 3 red) = [C(8, 3) * C(5, 0)] / C(13, 3)\\

P(X = 3 red) = [(56) * (1)] / (286)\\

P(X = 3 red) = 56 / 286 ≈ 0.195804\\   

(c)Now, let's calculate the total probability of getting one red and two white balls (which includes the cases where X = 1 red and 2 white and X = 2 red and 1 white):\\

Total probability of one red and two white = P(X = 1 red) + P(X = 2 red and 1 white)\\

Total probability = 0.279720 + [C(8, 2) * C(5, 1)] / C(13, 3)\\

                 = 0.279720 + [(28) * (5)] / (286)\\

                 = 0.279720 + 140 / 286\\

                 =0.279720 + 0.489510 = 0.769230
\end{align}


\end{document}

